% SOC TEMPLATE by KryKom
% Licencováno pro Tebe pod MIT Licencí

\documentclass{Template/Thesis}

% ================================= IMPORTY ================================= % 

\usepackage{amsmath}
\usepackage{caption}
\usepackage{csquotes}
\usepackage{enumitem}
\usepackage[T1]{fontenc}
\usepackage{geometry}
\usepackage{graphicx}
\usepackage{hyperref}
\usepackage{listings}
\usepackage{minted}
\usepackage{natbib}
\usepackage{newfloat}
\usepackage{parskip}
\usepackage{pgfkeys}
\usepackage{setspace}
\usepackage{subcaption}
\usepackage{tcolorbox}
\usepackage{tgtermes}
\usepackage{titlesec}
\usepackage[nottoc]{tocbibind}
\usepackage{xcolor}


% ================================= NASTAVENÍ ================================= % 

\tcbuselibrary{listings,skins}

% české uvozovky
\usepackage{csquotes}
\DeclareQuoteAlias{german}{czech}
\MakeOuterQuote{"}

% rozměry
\newgeometry{
    top = 2.5cm, 
    bottom = 2.5cm, 
    left = 2.5cm, 
    right = 2.5cm
}

% mezera mezi odstavci
\setlength{\parskip}{12pt}

% nadpisy
\titleformat{\chapter}{\fontsize{18pt}{21.6pt}\scshape\bfseries}{\thechapter}{18pt}{}
\titleformat{\section}{\fontsize{18pt}{21.6pt}\bfseries}{\thesection}{18pt}{}
\titleformat{\subsection}{\fontsize{14pt}{16.8pt}\bfseries}{\thesubsection}{14pt}{}

% nastavení popisků
\captionsetup{font = footnotesize}

% float kódu a rovnice
\DeclareFloatingEnvironment[fileext = lol]{code}
\DeclareFloatingEnvironment[fileext = lol]{rovnice}

% zobrazení popisku kódu
\captionsetup[code]{%
    name = Ukázka kódu,
}

% zobrazení popisku rovnice
\def\equationautorefname{Rovnice}

% nastavení umisťování tabulek
\makeatletter
\renewcommand{\fps@table}{h}
\makeatother


% ================================= INFO O PRÁCI ================================= % 

% přepiš podle vlastních potřeb
\titlecz{název práce}
\titleen{work name}
\author{tvoje jméno}
\field{18}
\school{Oficiální název, ulice č. p., PSČ město}
\mentor{mentor}
\mentorstatement{doc. PhDr. Jany Novákové, Ph.D.}
\region{kraj}
\city{město}
\created{rok vzniku}


% ================================= DOKUMENT ================================= % 

\begin{document}

% nějaký příkazy a nastavení
% SOC TEMPLATE by KryKom
% Licencováno pro Tebe pod MIT Licencí

% nastav si barvičky ty ricere, stejně se z nich používá jenom crefb a LatteBase

\definecolor{codegreen}{RGB}{64, 160, 43}
\definecolor{codegray}{RGB}{76, 79, 105}
\definecolor{codeorange}{RGB}{254, 100, 11}
\definecolor{codepurple}{RGB}{136, 57, 239}
\definecolor{backcolour}{RGB}{239, 241, 245}
\definecolor{linenumbers}{RGB}{114, 135, 253}
\definecolor{crefb}{RGB}{239, 241, 245}

\definecolor{LatteBase}{HTML}{EFF1F5}   % Light Background
\definecolor{LatteText}{HTML}{4C4F69}   % Dark Text (for normal code)
\definecolor{LatteMauve}{HTML}{8839EF}  % Keywords
\definecolor{LatteGreen}{HTML}{40A02B}  % Strings
\definecolor{LatteBlue}{HTML}{1E66F5}   % Functions
\definecolor{LattePeach}{HTML}{FE640B}  % Numbers
\definecolor{LatteGray}{HTML}{9CA0B0}   % Comments (Overlay0)
% SOC TEMPLATE by KryKom
% Licencováno pro Tebe pod MIT Licencí

% nastavení

\pgfkeys{
    /cinput/.is family, /cinput,
    caption/.store in=\ciCaption,
    label/.store in=\ciLabel,
    default/.style={caption=, label=}
}

\usemintedstyle{autumn}
\setminted{
    bgcolor=LatteBase,
    fontsize=\small,
    frame=single,
    framesep=2mm,
    rulecolor=\color{LatteBase},
    linenos
}

% příkazy

\newcommand{\B}[1]{\textbf{#1}}
\newcommand{\It}[1]{\textit{#1}}
\newcommand{\bs}{\textbackslash}

\newcommand{\importch}[1]{%
    \input{#1}
    \newpage
}

\newcommand{\cinline}[1]{%
    \tcbset{
  		on line,
  		boxsep=2pt, left=0pt,right=0pt,top=0pt,bottom=0pt,
  		colframe=white, colback=crefb
	}
	\tcbox{\ttfamily{\small{#1}}}
}

\newcommand{\cinput}[4]{%
    \begin{code}[h!]
        \centering
        \inputminted[label=#3]{#2}{#1}
        \caption{#4}
		\label{#3}
    \end{code}
}

\newcommand{\csnippet}[4]{%
    \begin{code}[h!]
        \centering
        \begin{minted}{#1}
            #4
        \end{minted}
        \caption{#3}
		\label{#2}
    \end{code}
}

\newcommand{\eq}[2]{%
    \begin{equation}
        #2
        \label{#1}
    \end{equation}
}

\newcommand{\img}[4]{%
    \begin{figure}[h!]
        \centering
        \includegraphics[scale = #2]{#1}
        \caption{#4}
        \label{#3}
    \end{figure}
}

\newcommand{\tbl}[4]{%
    \begin{table}[h!]
        \centering
        \caption{#2}
        \label{#1}
        \begin{tabular}{#3}
            #4
        \end{tabular}
    \end{table}
}

% Nastavení číslování prostředí "lstlistings".
\renewcommand{\thelstlisting}{\arabic{lstlisting}}

\maketitle


% ================================= LEGALNÍ BOBKY ================================= % 

% pokud jsi nepoužil(a) AI
\makecopyrightstatement{V~Praze}{\ainotused}

% pokud jsi použil(a) AI (například, dej tam co chceš, ale odkomentuj to :D)
% \makecopyrightstatement{V~Praze}{\aiused{gemini.google.com}{reformulace textu}}


% ================================= PODĚKOVÁNÍ ================================= % 

\makethanks{
    %sem vlož poděkování
}

\pagestyle{empty}


% ================================= ANOTACE ================================= % 

\section*{Anotace}

% sem vlož svoji anotaci


\section*{Klíčová slova}

% sem vlož klíčová slova oddělená středníkem


\vspace{20mm}

\section*{Annotation}

% sem vlož svoji anotaci v angličtině


\section*{Keywords}

% sem vlož klíčová slova v angličtině oddělená středníkem


\newpage
\pagestyle{plain}


% Obsah

\titleformat{\section}{\fontsize{16pt}{19.2pt}\bfseries}{\thesection}{16pt}{}

\tableofcontents

\setcounter{figure}{0}
\setcounter{table}{0}
\newpage

% ================================= TA SAMOTNÁ PRÁCE ================================= %

\importch{Intro.tex}
\importch{Kapitola.tex}

% =================================       KONEC      ================================= %

\listoffigures
\listoftables

\renewcommand\lstlistlistingname{Ukázky kódu}
\lstlistoflistings
\addcontentsline{toc}{chapter}{Ukázky kódu}

\end{document}