\newpage

\chapter{Kapitola}
\label{ch:kapitola}

Tato kapitola popisuje jak vkládat jednotlivé prvky do dokumentu.
Prvním prvkem je nadpis kapitoly, jenž lze vložit pomocí \cinline{\bs chapter}.
Dovnitř vložíte název kapitoly.

\section{Podnadpis}

Nadpis podkapitoly lze vložit pomocí příkazu \cinline{\bs section}, 
do kterého jako parametr vložíte název podkapitoly.

\subsection{Podpodnadpis}

Funguje to podobně jako u podkapitoly. 
Nadpis vytvoříte pomocí \cinline{\bs subsection}.

Další levely kapitol se nepoužívají.

\section{Obrázky}

Obrázky lze vkládat pomocí příkazu \cinline{\bs img}.
První parametr je cesta k obrázku, druhý je velikost obrázku, 
třetí je label a čtvrtý popisek. Referenci na obrázek vytvoříte
pomocí příkazu \cinline{\bs autoref}, do kterého vložíte label obrázku. 
Příkladem je \autoref{img:opice}.

\img{Assets/Image.png}{0.08}{img:opice}{Opice}

\section{Poznámky pod čarou}

Poznámku pod čarou vložíte pomocí příkazu \cinline{\bs footnote}.
Do něj vložíte samotnou poznámku. Například:

V tomhle textu je potřeba něco\footnote{nebo taky ne} vysvětlit...

\section{Tabulky}

V \TeX u mají tabulky dost podivnou syntax. Snažil jsem se ji trošku
zjednodušit pomocí příkazu \cinline{\bs tbl}. První parametr obsahuje 
label, druhý popisek, třetí definici sloupců a čtvrtý data.

Data se vkládají poněkud podivně. Například řádky se oddělují pomocí
\cinline{\bs \bs ~\bs hline} (\cinline{\bs hline} vytvoří horizontální čáru).
Jednotlivá políčka se oddělují pomocí \cinline{\&}.

Definici tabulky \autoref{t:tbl} si můžeme prohlédnout na 
výstřižku \autoref{c:table}.

\tbl{t:tbl}{Tabulka}{|l|l|}{%
    \hline
    \HT{Sloupec 0} & \HT{Sloupec 1} \\ \hline
    pole 0         & pole 1         \\ \hline
}

\cinput{Snippets/table.tex}{tex}{c:table}{Definice tabulky}

A tady níže lze spatřit jinak stylovanou tabulku \autoref{t:tabulka}.

\tbl{t:tabulka}{Pozdrav}{l||l}{
    \B{Hello} & \B{World} \\ \hline
    Ahoj    & z       \\ \hline
    světe   & této    \\ \hline
    zdravím & tabulky \\
}

\section{Kód}

Pomocí \cinline{\bs cinline} můžeme doprostřed textu vkládat znaky kódu. 
Pokud ale chceme vložit větší kus kódu učiníme tak pomocí \cinline{\bs cinput}.
První parametr je cesta k souboru, druhý je jazyk, třetí label a čtvrtý popisek.
Příklad využití můžeme vidět na ukázce \autoref{c:cinput} \footnote{je to rekurzivní :)}
nebo také \autoref{c:hovno}.

\cinput{Snippets/cinput.tex}{tex}{c:cinput}{Použití cinput}

\cinput{Snippets/is3.py}{python}{c:hovno}{Hovno}

% Dalším způsobem jak vkládat kód je pomocí \cinline{\bs csnippet}.
% První parametr je jazyk, druhý je label, třetí popisek a čtvrtý je 
% kód samotný. Příklad využití můžete vidět na ukázce 

% \csnippet{csharp}{c:csharp}{Csharp}{%
%     Console.WriteLine\(\"Hello!\"\)\;
% }

\section{Rovnice}

Dále také lze vkládat rovnice pomocí \cinline{\bs eq}. 
Příkladem může být \autoref{e:wut} jejíž definici lze 
spatřit na ukázce \autoref{c:eq}.

\eq{e:wut}{2 + 2 = 5 \wedge 2 + 2 = 3}

\cinput{Snippets/eq.tex}{tex}{c:eq}{Definice rovnice z 1984}

\section{Citace}

Pokud chceme citovate nějaký zdroj použijeme \cinline{\bs cite}.
Například pokud bych chtěl citovat myšlenku z 1984, že 
někdy se 2 a 2 rovná 5 a někdy zase 3 \cite{nef}, napíšu za myšlenku 
\cinline{\bs cite\{nef\}} (\cinline{nef} se jmenuje 1984 v 
\cinline{Ref.bib}). Zdroj v \cinline{Ref.bib} je definován stejně jako na
ukázce \autoref{c:bib}. Se syntaxí zdrojů poradí \LaTeX ~Workshop ve VS Code.

\cinput{Snippets/1984.bib}{bib}{c:bib}{1984 jako zdroj}

\newpage