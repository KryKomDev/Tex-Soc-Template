\newpage

\chapter{Úvod}

Ahoj a vítej v tomto úvodu do této šablony pro práce SOČ.
Začátkem bych chtěl podotknout, že doufám, že ti tato šablona pomůže 
a zjednoduší ti práci s \LaTeX em. V případě, že jsem ti pomohl,
bych tě také chtěl poprosit, jestli bys mi nedal(a) na repo 
hvězdičku, jako takové poděkování.

V této jen v rychlosti představím, jak tuto šablonu používat a 
jak nastavovat různé věci. Jak vkládat různé tabulky, obrázky a podobně
se dovíte až v kapitole \autoref{ch:kapitola}.

\section{Main.tex}

Hlavním souborem dokumentu je \cinline{Main.tex}. Ten obsahuje float
dokumentu a do něj také budete vkládat veškeré kapitoly (jako například 
\cinline{Intro.tex} nebo \cinline{Kapitola.tex}). Místo na vkládání vypadá
asi tak, jak je zobrazeno na ukázce \autoref{c:chapters}. Kapitolu lze
vložit poměrně jednoduše pomocí příkazu \cinline{\textbackslash importch}, 
do kterého jako parametr vložíte název souboru, ve kterém je kapitola
uložena.

\cinput{Snippets/chapters.tex}{tex}{c:chapters}{Místo pro vkládání kapitol}

V \cinline{Main.tex} se také nachází nastavení informací o práci. To 
je zobrazeno na ukázce \autoref{c:workinfo}. 

\cinput{Snippets/workinfo.tex}{tex}{c:workinfo}{Místo pro nastavení informací o práci}

Hned za touto částí se nachází začátek dokumentu. Ještě o pár řádků dál
se nachází prohlášení o všech možných věcech (přečtěte si to!!!).
Tam si nastavíte jestli jste používali nebo nepoužívali AI (pokud ano tak 
jaký model a za jakým účelem).

Pak už hned následuje poděkování a anotace a potom je 
už zmíněné místo na vkládání kapitol.

\section{Složka Template}

\cinline{Thesis.cls} definuje příkazy pro šablonu ale také 
nastavuje zobrazování referencí v textu.

\cinline{Commands.tex} obsahuje užitečné příkazy jako je například
\cinline{\bs cinline} pro bloky kódu uvnitř textu, 
\cinline{\bs cinput} pro vkládání kódu ze souborů, 
\cinline{\bs csnippet} pro vkládání kódu, 
\cinline{\bs bs} pro vložení zpětného lomítka, 
\cinline{\bs B} pro tučný text a také 
\cinline{\bs It} pro text psaný kurzívou.

\cinline{Colors.tex} obsahuje definice několika používaných barev.

\section{Ref.bib}

\cinline{Ref.bib} je seznam literatury. Více o tom jak vkládat
zdroje se dozvíte například \href{https://cs.overleaf.com/learn/latex/Bibliography\_management\_with\_biblatex}{zde}.

\newpage